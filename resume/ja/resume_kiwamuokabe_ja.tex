% LaTeX Curriculum Vitae Template
%
% Copyright (C) 2004-2009 Jason Blevins <jrblevin@sdf.lonestar.org>
% http://jblevins.org/projects/cv-template/
%
% You may use use this document as a template to create your own CV
% and you may redistribute the source code freely. No attribution is
% required in any resulting documents. I do ask that you please leave
% this notice and the above URL in the source code if you choose to
% redistribute this file.

\documentclass[letterpaper]{article}

\usepackage[dvipdfmx]{graphicx}
\usepackage{fancybox}
\usepackage{longtable}
\usepackage{fancyvrb}
\usepackage[dvipdfmx]{hyperref}
\usepackage{url}
\usepackage[dvipdfmx]{color}
\usepackage{nextpage}
\usepackage{float}
\usepackage{wrapfig}
\usepackage{tabularx}
\usepackage{multicol}
\usepackage{makeidx}
\usepackage{ascmac} % screen
\usepackage[dvips]{xy} % for advi workaround. Bug #452044

\usepackage{geometry}

% Comment the following lines to use the default Computer Modern font
% instead of the Palatino font provided by the mathpazo package.
% Remove the 'osf' bit if you don't like the old style figures.
%\usepackage[T1]{fontenc}
%\usepackage[sc,osf]{mathpazo}

% Set your name here
\def\name{Kiwamu Okabe (岡部 究)}

% Replace this with a link to your CV if you like, or set it empty
% (as in \def\footerlink{}) to remove the link in the footer:
\def\footer{
  \begin{center}
    \begin{footnotesize}
      Last updated: \today
    \end{footnotesize}
  \end{center}
}

% The following metadata will show up in the PDF properties
\hypersetup{
  colorlinks = true,
  urlcolor = black,
  pdfauthor = {\name},
  pdfkeywords = {economics, statistics, mathematics},
  pdftitle = {\name: Curriculum Vitae},
  pdfsubject = {Curriculum Vitae},
  pdfpagemode = UseNone
}

\geometry{
  body={6.5in, 8.5in},
  left=1.0in,
  top=1.25in
}

% Customize page headers
\pagestyle{myheadings}
\markright{\name}
\thispagestyle{empty}

% Custom section fonts
\usepackage{sectsty}
\sectionfont{\rmfamily\mdseries\Large}
\subsectionfont{\rmfamily\mdseries\itshape\large}

% Other possible font commands include:
% \ttfamily for teletype,
% \sffamily for sans serif,
% \bfseries for bold,
% \scshape for small caps,
% \normalsize, \large, \Large, \LARGE sizes.

% Don't indent paragraphs.
\setlength\parindent{0.2em}

%% Make lists without bullets
%\renewenvironment{itemize}{
%  \begin{list}{}{
%    \setlength{\leftmargin}{1.5em}
%  }
%}{
%  \end{list}
%}

\begin{document}

% Place name at left
{\huge \name}

% Alternatively, print name centered and bold:
%\centerline{\huge \bf \name}

\vspace{0.25in}

\begin{minipage}{0.3\linewidth}
  \begin{tabular}{ll}
    Phone: & +81-90-3524-7064 \\
    Email: & \href{mailto:kiwamu@debian.or.jp}{\tt kiwamu@debian.or.jp} \\
    Homepage: & \href{http://masterq.metasepi-design.com/}{\tt http://masterq.metasepi-design.com/} \\
  \end{tabular}
\end{minipage}

\section*{Brief}

私は株式会社リコーにて10年のUnixライクカーネルを用いた組み込み開発に従事しました。それは単なるデバイスドライバの開発だけではなく仮想メモリのデバッグにおよぶ幅広い経験でした。またARM Cortex-M MCUやRTOSのようなIoTプラットフォームに対する経験も保有しています。さらにHaskell\footnote{\href{https://www.haskell.org/}{\tt https://www.haskell.org/}}のような関数型プログラミング言語を持ちいたアプリケーション設計も学び、その成果を論文化\footnote{\href{http://www.metasepi.org/papers.html}{\tt http://www.metasepi.org/papers.html}}もしてきました。

これからもソフトウェアのレイヤーにかかわらず(セキュリティを含む)品質向上に関する技術の追求を続けたいと考えています。

\section*{Skill Set}

\subsection*{Unixライクカーネルとユーザ空間に対する深い知識}

私はLinuxのようなUnixライクカーネルのエキスパートであり、リコーではLinuxと良く似たUnixライクOSであるNetBSD\footnote{\href{http://netbsd.org/}{\tt http://netbsd.org/}}に対するテクニカルサポートを提供していました。このスキルはNetBSDのみならずLinuxに関しても同様です。その証拠にミラクル・リナックスにてPowerPC Linuxのレースコンディションバグはたった5日間の内に私が発見/修復しました。

\subsection*{IoTプラットフォームにおける幅広い経験}

私はFreeRTOSやChibiOS/RT\footnote{\href{http://www.chibios.org/}{\tt http://www.chibios.org/}}、ARM Cortex-M、ESP8266、AVR、MSP430のようなIoTプラットフォームに幅広い経験があります。またセンティリオンではTWELITEワイヤレスプラットフォーム\footnote{\href{https://mono-wireless.com/}{\tt https://mono-wireless.com/}}を用いて新しいIoTビジネスを立ち上げました。

\subsection*{セキュリティと品質の探求}

SELTECH社で私はOP-TEE\footnote{\href{https://github.com/OP-TEE/optee\_os}{\tt https://github.com/OP-TEE/optee\_os}}と良く似た独自セキュアOSのプロトタイプを設計しました。そのSecure-OSはARM Cortex-M MCUの上で別のRTOSと協調動作します。また私はHaskellのような強い型システムを持つ言語とVeriFast\footnote{\href{https://github.com/verifast/verifast}{\tt https://github.com/verifast/verifast}}のようなC言語の検証について広い経験があります。それらは製品の品質を維持するのに有用です。

\newpage

\section*{Work Experience}

%%%%%%%%%%%%%%%%
\subsection*{2020年12月 - 現在: 株式会社Donutsにてソフトウェアエンジニア(正社員)}

\noindent Responsibilities:

\begin{itemize}
  \item ドキュメント不在の古いPHP製Webアプリケーションのメンテナンスとパフォーマンス改善
\end{itemize}

\noindent Key Achievements:

\begin{itemize}
  \item パフォーマンス改善策に対するテスト手法の立案
\end{itemize}

%%%%%%%%%%%%%%%%
\subsection*{2013年07月 - 2020年11月: フリーランス研究者}

\noindent Responsibilities:

\begin{itemize}
  \item Ajhc Haskellコンパイラ\footnote{\href{http://ajhc.metasepi.org/}{\tt http://ajhc.metasepi.org/}}の研究開発
  \item 組み込み開発におけるATS言語エバンジェリスト\footnote{\href{http://www.ats-lang.org/}{\tt http://www.ats-lang.org/}}
  \item 事前条件と事後条件を付記したC言語プログラムの検証器であるVeriFast\footnote{\href{https://github.com/verifast/verifast}{\tt https://github.com/verifast/verifast}}を用いたソフトウェア検証のエバンジェリスト
\end{itemize}

\noindent Key Achievements:

\begin{itemize}
  \item 論文の公開\footnote{\href{http://www.metasepi.org/papers.html}{\tt http://www.metasepi.org/papers.html}}
  \item ATSドキュメントの日本語翻訳\footnote{\href{http://jats-ug.metasepi.org/}{\tt http://jats-ug.metasepi.org/}}
  \item VeriFastチュートリアルの日本語翻訳\footnote{\href{https://github.com/jverifast-ug/translate/blob/master/Manual/Tutorial/Tutorial.md}{\tt https://github.com/jverifast-ug/translate/blob/master/Manual/Tutorial/Tutorial.md}}
\end{itemize}

%%%%%%%%%%%%%%%%
\subsection*{2019年10月 - 2020年03月: 株式会社QuantumCore にてソフトウェアエンジニア(自営業契約)}

\noindent Responsibilities:

\begin{itemize}
  \item 機械学習アルゴリズム(リザーバコンピューティング)のARM Cortex-Mプロセッサへの移植
  \item 機械学習アルゴリズム(リザーバコンピューティング)のAndroidへの移植
\end{itemize}

\noindent Key Achievements:

\begin{itemize}
  \item ARM Cortex-Mプロセッサ向け線形代数ライブラリの開発
\end{itemize}

%%%%%%%%%%%%%%%%
\subsection*{2018年2月 - 2018年07月: 株式会社新川にてソフトウェア・アーキティクト(契約社員)}

\noindent Responsibilities:

\begin{itemize}
  \item 半導体製造におけるワイヤ・ボンディングのための新しいソフトウェアプラットフォームの研究開発
\end{itemize}

\noindent Key Achievements:

\begin{itemize}
  \item 新川独自の組み込みスクリプト言語を解釈するパーサを作成
  \item EtherCAT\footnote{\href{https://www.ethercat.org/}{\tt https://www.ethercat.org/}}プロトコルのリアルタイム性評価
\end{itemize}

%%%%%%%%%%%%%%%%
\subsection*{2014年08月 - 2017年10月: 独立行政法人理化学研究所 計算科学研究機構にて研究嘱託}

\noindent Responsibilities:

\begin{itemize}
  \item 組み込みプラットフォーム向け関数型プログラミングの研究開発
\end{itemize}

\noindent Key Achievements:

\begin{itemize}
  \item 論文の公開\footnote{\href{http://www.metasepi.org/papers.html}{\tt http://www.metasepi.org/papers.html}}
\end{itemize}

%%%%%%%%%%%%%%%%
\subsection*{2016年11月 - 2017年10月: 株式会社SELTECHにてエキスパートエンジニア(正社員)}

\noindent Responsibilities:

\begin{itemize}
  \item ARM Cortex-Mプラットフォーム向けSecure OSの新規開発
\end{itemize}

%%%%%%%%%%%%%%%%
\subsection*{2016年02月 - 2016年11月: ライフロボティクス株式会社にてソフトウェアエンジニア(契約社員)}

\noindent Responsibilities:

\begin{itemize}
  \item Ubuntu OS, C++, Qt\footnote{\href{https://www.qt.io/}{\tt https://www.qt.io/}}を用いたシングルアームロボットのためのGUIアプリケーションの設計
\end{itemize}

\noindent Key Achievements:

\begin{itemize}
  \item ロボット向けネットワークプロトコルの策定
\end{itemize}

%%%%%%%%%%%%%%%%
\subsection*{2015年03月 - 2016年02月: センティリオン株式会社にてソフトウェアエンジニア(契約社員)}

\noindent Responsibilities:

\begin{itemize}
  \item JavaScriptを使った株式チャートアプリケーションに対する技術サポート
  \item MySQLデータベースサーバの保守運用
  \item 中国のオフショア開発のテクニカルマネジメント
\end{itemize}

\noindent Key Achievements:

\begin{itemize}
  \item 農業向けIoTビジネスの立ち上げ
  \item R言語\footnote{\href{https://www.r-project.org/}{\tt https://www.r-project.org/}}プログラムを高速化するプラットフォーム\footnote{\href{https://github.com/centillion-tech/kick-r}{\tt https://github.com/centillion-tech/kick-r}}の作成
\end{itemize}

%%%%%%%%%%%%%%%%
\subsection*{2014年09月 - 2014年12月: 株式会社あくしゅにてソフトウェアエンジニア(自営業契約)}

\noindent Responsibilities:

\begin{itemize}
  \item OpenFlowアプリケーション ``OpenVNet''\footnote{\href{https://github.com/axsh/openvnet}{\tt https://github.com/axsh/openvnet}} の開発
\end{itemize}

\noindent Key Achievements:

\begin{itemize}
  \item RubyとGNU makeを用いたOpenVNetのAWSへのデプロイ自動化
\end{itemize}

%%%%%%%%%%%%%%%%
\subsection*{2012年03月 - 2013年07月: ミラクル・リナックス株式会社にてソフトウェアエンジニア(正社員)}

\noindent Responsibilities:

\begin{itemize}
  \item Linux OS, C++, OpenGL, GTK+\footnote{\href{https://www.gtk.org/}{\tt https://www.gtk.org/}}, GStreamer\footnote{\href{https://gstreamer.freedesktop.org/}{\tt https://gstreamer.freedesktop.org/}}を用いたIntelアーキティクチャ上での独自デジタルサイネージプラットフォームの開発
  \item 独自Linuxディストリビューションのサポートとデバッグ
\end{itemize}

\noindent Key Achievements:

\begin{itemize}
  \item PowerPC Linux kernelにおけるレースコンディションの発見/修正
  \item crash\footnote{\href{http://people.redhat.com/{\textasciitilde}anderson/}{\tt http://people.redhat.com/{\textasciitilde}anderson/}}コマンドのPowerPC仮想メモリの不具合を発見/修正
  \item NSIS\footnote{\href{http://nsis.sourceforge.net/}{\tt http://nsis.sourceforge.net/}}を使った新しいWindowsアプリケーションインストーラの作成
\end{itemize}

%%%%%%%%%%%%%%%%
\subsection*{2001年04月 - 2012年02月: 株式会社リコーにてソフトウェアエンジニア(正社員)}

\noindent Responsibilities:

\begin{itemize}
  \item NetBSD OSを使った複合機向けソフトウェアプラットフォームの開発
\end{itemize}

\noindent Key Achievements:

\begin{itemize}
  \item Intelアーキティクチャ上のOptionBIOSとbootloaderを開発
  \item Intelアーキティクチャにおけるセキュアブートの実現
  \item 複合機の起動時間を10秒に圧縮
  \item M対N POSIXスレッドライブラリの検証
\end{itemize}

\section*{Education}

\begin{itemize}
  \item 2001年03月: 東京都立大学 修士卒業/工学研究科電気工学専攻 電気・電子工学 \\
    研究概要: 水晶振動子のマルチモードを使用した気体センサの作成\footnote{\href{http://ci.nii.ac.jp/naid/110004076869}{\tt http://ci.nii.ac.jp/naid/110004076869}}
\end{itemize}

\section*{Publications and Reports}

\begin{itemize}
  \item 岡部究, Hongwei Xi 「Arduino programing of ML-style in ATS」, ML workshop 2015\footnote{\href{http://www.metasepi.org/doc/metasepi-icfp2015-arduino-ats.pdf}{\tt http://www.metasepi.org/doc/metasepi-icfp2015-arduino-ats.pdf}}
  \item 岡部究, 村主崇行 「Systems Demonstration: Writing NetBSD Sound Drivers in Haskell」, Haskell Symposium 2014\footnote{\href{http://metasepi.org/doc/metasepi-icfp2014-demo.pdf}{\tt http://metasepi.org/doc/metasepi-icfp2014-demo.pdf}}
  \item 2014年08月: 「ATS言語を使って不変条件をAPIに強制する」, 夏のプログラミング・シンポジウム 2014\footnote{\href{http://www.metasepi.org/doc/20141101\_prosym\_summer2014.pdf}{\tt http://www.metasepi.org/doc/20141101\_prosym\_summer2014.pdf}}  2014.
  \item 2014年01月: 「強い型によるOSの開発手法の提案」, 第55回プログラミング・シンポジウム\footnote{\href{http://metasepi.org/doc/20140110\_prosym55.pdf}{\tt http://metasepi.org/doc/20140110\_prosym55.pdf}}
\end{itemize}

\section*{Activities}

\subsection*{Open-source projects}

\subsubsection*{Metasepi project\footnote{\href{http://metasepi.org/}{\tt http://metasepi.org/}}}
\begin{itemize}
\item MLやHaskellのような強い型を使ってUNIXライクkernelを書き直すプロジェクト。現在NetBSD kernelのドライバをHaskell化中。 \href{https://github.com/metasepi/netbsd-arafura-s1}{\tt https://github.com/metasepi/netbsd-arafura-s1}
\end{itemize}

\subsubsection*{Ajhc Haskell compiler\footnote{\href{http://ajhc.metasepi.org/}{\tt http://ajhc.metasepi.org/}}}
\begin{itemize}
\item Jhc Haskell Compiler\href{http://repetae.net/computer/jhc/}{\tt http://repetae.net/computer/jhc/} に組み込み拡張を加えたHaskellコンパイラ。再入可能なプログラムを作成でき、メモリ数十kBでも動作可能なバイナリを吐く。
\end{itemize}

\subsubsection*{Japan ATS User Group\footnote{\href{http://jats-ug.metasepi.org/}{\tt http://jats-ug.metasepi.org/}}}
\begin{itemize}
\item 日本におけるATS言語\href{http://www.ats-lang.org/}{\tt http://www.ats-lang.org/} の利用促進を目的としたユーザグループ。ATS関連ドキュメントを日本語訳中。
\end{itemize}

\subsubsection*{Debian Maintainer\footnote{\href{http://qa.debian.org/developer.php?login=kiwamu@debian.or.jp}{\tt http://qa.debian.org/developer.php?login=kiwamu@debian.or.jp}}}
\begin{itemize}
\item Debian squeezeにてuim、sidにてHaskell関連パッケージメンテナ。
\end{itemize}

\section*{Computer Skills}

\begin{itemize}
  \item Languages: C, C++, Haskell, Intel/ARM assembler, Ruby, PHP, SQL, OCaml, Python, Erlang, JavaScript, R
  \item Platforms: Linux, NetBSD, FreeRTOS, ChibiOS/RT, Android NDK, Cygwin, MinGW, Bare metal
\end{itemize}

\section*{Reference available upon request}

\begin{itemize}
  \item 秋吉 信吾 CEO - 株式会社QuantumCore
  \item 宗像 広志 CTO - 株式会社新川
  \item 江川 将偉 代表取締役社長 - 株式会社SELTECH
  \item 尹 祐根 代表取締役 - ライフロボティクス株式会社
  \item 黒岩 健太郎 研究主任 - センティリオン株式会社
  \item 山崎 泰宏 代表取締役 - 株式会社あくしゅ
  \item 牧野 淳一郎 チームリーダー - 独立行政法人理化学研究所 計算科学研究機構
  \item 児玉 崇 社長 - ミラクル-リナックス株式会社
  \item 千田 滋也 - 株式会社リコー
  \item 関本 仁 教授 - 首都大学東京 工学部電気工学科 電気・電子工学
\end{itemize}

\bigskip
\footer

\end{document}
