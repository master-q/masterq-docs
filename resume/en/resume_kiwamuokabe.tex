% LaTeX Curriculum Vitae Template
%
% Copyright (C) 2004-2009 Jason Blevins <jrblevin@sdf.lonestar.org>
% http://jblevins.org/projects/cv-template/
%
% You may use use this document as a template to create your own CV
% and you may redistribute the source code freely. No attribution is
% required in any resulting documents. I do ask that you please leave
% this notice and the above URL in the source code if you choose to
% redistribute this file.

\documentclass[letterpaper]{article}

\usepackage[dvipdfmx]{graphicx}
\usepackage{fancybox}
\usepackage{longtable}
\usepackage{fancyvrb}
\usepackage[dvipdfmx]{hyperref}
\usepackage{url}
\usepackage[dvipdfmx]{color}
\usepackage{nextpage}
\usepackage{float}
\usepackage{wrapfig}
\usepackage{tabularx}
\usepackage{multicol}
\usepackage{makeidx}
\usepackage{ascmac} % screen
\usepackage[dvips]{xy} % for advi workaround. Bug #452044

\usepackage{geometry}

% Comment the following lines to use the default Computer Modern font
% instead of the Palatino font provided by the mathpazo package.
% Remove the 'osf' bit if you don't like the old style figures.
%\usepackage[T1]{fontenc}
%\usepackage[sc,osf]{mathpazo}

% Set your name here
\def\name{Kiwamu Okabe - Fullstack Engineer}

% Replace this with a link to your CV if you like, or set it empty
% (as in \def\footerlink{}) to remove the link in the footer:
\def\footer{
  \begin{center}
    \begin{footnotesize}
      Last updated: \today
    \end{footnotesize}
  \end{center}
}

% The following metadata will show up in the PDF properties
\hypersetup{
  colorlinks = true,
  urlcolor = black,
  pdfauthor = {\name},
  pdfkeywords = {economics, statistics, mathematics},
  pdftitle = {\name: Curriculum Vitae},
  pdfsubject = {Curriculum Vitae},
  pdfpagemode = UseNone
}

\geometry{
  body={6.5in, 8.5in},
  left=1.0in,
  top=1.25in
}

% Customize page headers
\pagestyle{myheadings}
\markright{\name}
\thispagestyle{empty}

% Custom section fonts
\usepackage{sectsty}
\sectionfont{\rmfamily\mdseries\Large}
\subsectionfont{\rmfamily\mdseries\itshape\large}

% Other possible font commands include:
% \ttfamily for teletype,
% \sffamily for sans serif,
% \bfseries for bold,
% \scshape for small caps,
% \normalsize, \large, \Large, \LARGE sizes.

% Don't indent paragraphs.
\setlength\parindent{0.2em}

%% Make lists without bullets
%\renewenvironment{itemize}{
%  \begin{list}{}{
%    \setlength{\leftmargin}{1.5em}
%  }
%}{
%  \end{list}
%}

\begin{document}

% Place name at left
{\huge \name}

% Alternatively, print name centered and bold:
%\centerline{\huge \bf \name}

\vspace{0.25in}

\begin{minipage}{0.3\linewidth}
  \begin{tabular}{ll}
    Phone: & +81-90-3524-7064 \\
    Email: & \href{mailto:kiwamu@gmail.com}{\tt kiwamu@gmail.com} \\
    Homepage: & \href{http://masterq.metasepi-design.com/}{\tt http://masterq.metasepi-design.com/} \\
  \end{tabular}
\end{minipage}

\section*{Brief}

I launched my career on developing embedded devices using Unix-like kernel at Ricoh Company, Ltd. The experience was not only for designing device drivers but also including debug of virtual memory. And I have experience in IoT platfom such as ARM Cortex-M MCU and RTOS. Also I learned application design using functional language such as Haskell\footnote{\href{https://www.haskell.org/}{\tt https://www.haskell.org/}}, and published some research papers\footnote{\href{http://www.metasepi.org/papers.html}{\tt http://www.metasepi.org/papers.html}} about such languages.

I would like to continue to pursue quality improvement technologies (including security) regardless of the software layer.

\section*{Skill Sets}

\subsection*{Deep knowledge for Unix-like Kernel and User Space}

I am an expert for Unix-like kernel such as Linux, because I provided technical support for NetBSD\footnote{\href{http://netbsd.org/}{\tt http://netbsd.org/}}, which is a Unix-like OS similar to Linux, at Ricoh. My skill is not only for NetBSD but also Linux. In fact, a race condition bug in PowerPC Linux kernel was fixed by me in only five days at MIRACLE LINUX.

\subsection*{Wide experience in IoT platform}

I have a wide experience in IoT platform such as FreeRTOS, ChibiOS/RT\footnote{\href{http://www.chibios.org/}{\tt http://www.chibios.org/}}, ARM Cortex-M, ESP8266, AVR and MSP430. Also I launched a new IoT business using TWELITE wireless platform\footnote{\href{https://mono-wireless.com/}{\tt https://mono-wireless.com/}} at Centillion Japan.

\subsection*{Research Security and Quality}

A prototype of own Secure-OS similar to OP-TEE\footnote{\href{https://github.com/OP-TEE/optee\_os}{\tt https://github.com/OP-TEE/optee\_os}} was designed by me at SELTECH. It runs with the other RTOS on ARM Cortex-M MCU. Also I have a wealth of experience in strong static typing language such as Haskell and verification of C language such as VeriFast\footnote{\href{https://github.com/verifast/verifast}{\tt https://github.com/verifast/verifast}}, which are useful to keep the quality of products.

\section*{Computer Skills}

\begin{itemize}
  \item Languages: C (12 years), Haskell (5 years), Intel/ARM assembler (5 years), Ruby (5 years), C++ (3 years), PHP (2 years), OCaml (2 years), SQL (1.5 years), Python (1 year), Erlang (1 year), JavaScript (1 year), R (1 year), Go (0.5 years)
  \item Platforms: Linux (15 years), NetBSD (12 years), Cygwin (2 years), FreeRTOS (1.5 years), ChibiOS/RT (1.5 years), Android NDK (1 year), MinGW (1 year), Yocto (1 year)
  \item Database: MySQL (2 years)
\end{itemize}

\newpage

\section*{Work Experience}

%%%%%%%%%%%%%%%%
\subsection*{May 2021 - Present: Freelance Researcher}

\noindent Responsibilities:

\begin{itemize}
  \item Improving the quality of open source OS
\end{itemize}

\noindent Key Achievements:

\begin{itemize}
  \item Found the root causes of FreeBSD OS bugs and vulnerabilities with SRE postmortem style, and avoided them with ATS and VeriFast\footnote{\href{https://github.com/metasepi/postmortem}{\tt https://github.com/metasepi/postmortem}}
\end{itemize}

%%%%%%%%%%%%%%%%
\subsection*{February 2023 - September 2023: Cloud Infrastructure Expert (permanent employee) at VA Linux Systems Japan K.K.}

\noindent Responsibilities:

\begin{itemize}
  \item Technical support for Linux distribution and open-source software
\end{itemize}

\noindent Key Achievements:

\begin{itemize}
  \item Verified Linux kernel with Infer static analyzer\footnote{\href{https://valinux-hatenablog-com.translate.goog/entry/20230803?_x_tr_sl=auto&_x_tr_tl=en}{\tt https://valinux-hatenablog-com.translate.goog/entry/20230803}}
\end{itemize}

\noindent Reason for changing job:

\begin{itemize}
  \item To focus developing embedded Rust application
\end{itemize}

%%%%%%%%%%%%%%%%
\subsection*{August 2022 - January 2023: Systems Engineer (permanent employee) at Systemi.co.ltd}

\noindent Responsibilities:

\begin{itemize}
  \item Entrusted development of web applications
\end{itemize}

\noindent Key Achievements:

\begin{itemize}
  \item Published a payment application running on Android tablet designed with Kotlin language
\end{itemize}

\noindent Reason for changing job:

\begin{itemize}
  \item Because I wanted to contribute to projects with a long-term perspective rather than short-term work such as contracted development
  \item Because I realized that embedded development is what I am suited for, not web server-side or Android application development
\end{itemize}

%%%%%%%%%%%%%%%%
\subsection*{August 2021 - February 2022: Systems \& Applications Engineer  (permanent employee) at NXP Japan Ltd.}

\noindent Responsibilities:

\begin{itemize}
  \item Technical support for NXP Microprocessors
\end{itemize}

\noindent Key Achievements:

\begin{itemize}
  \item Supported audio application using Yocto Linux and Android platform
\end{itemize}

\noindent Reason for changing job:

\begin{itemize}
  \item Because I was engaged primarily in customer support and did not have the authority to commit to source code, I did not realize the progress and contribution I was working daily
\end{itemize}

%%%%%%%%%%%%%%%%
\subsection*{December 2020 - April 2021: Software Engineer (permanent employee) at Donuts Co. Ltd.}

\noindent Responsibilities:

\begin{itemize}
  \item Maintained an ERP web application using PHP, Zend Framework, JavaScript, MySQL, and AWS
\end{itemize}

\noindent Key Achievements:

\begin{itemize}
  \item Created a black-box testing tool running Docker to get better performance and keep quality
  \item Created a summarizer of MySQL query log using Go language
\end{itemize}

\noindent Reason for changing job:

\begin{itemize}
  \item The sheer complexity of the business logic and the absence of design documentation made the code extremely difficult to be maintained
\end{itemize}

%%%%%%%%%%%%%%%%
\subsection*{July 2013 - November 2020: Freelance Researcher}

\noindent Responsibilities:

\begin{itemize}
  \item Researched and developed Ajhc Haskell Compiler\footnote{\href{http://ajhc.metasepi.org/}{\tt http://ajhc.metasepi.org/}}
  \item ATS language\footnote{\href{http://www.ats-lang.org/}{\tt http://www.ats-lang.org/}} evangelist for embedded devices
  \item Verification evangelist using VeriFast\footnote{\href{https://github.com/verifast/verifast}{\tt https://github.com/verifast/verifast}}, which is a verifier C language programs annotated with preconditions and postconditions
\end{itemize}

\noindent Key Achievements:

\begin{itemize}
  \item Published some research papers\footnote{\href{http://www.metasepi.org/papers.html}{\tt http://www.metasepi.org/papers.html}}
  \item Translated ATS documents into Japanese\footnote{\href{http://jats-ug.metasepi.org/}{\tt http://jats-ug.metasepi.org/}}
  \item Translated VeriFast documents into Japanese\footnote{\href{https://github.com/jverifast-ug/translate}{\tt https://github.com/jverifast-ug/translate}}
\end{itemize}

%%%%%%%%%%%%%%%%
\subsection*{October 2019 - March 2020: Software Engineer (trustee agreement) at QuantumCore CORPORATION}

\noindent Responsibilities:

\begin{itemize}
  \item Ported a machine learning called ``reservoir computing'' onto ARM Cortex-M MCU
  \item Ported the machine learning onto Android platform
\end{itemize}

\noindent Key Achievements:

\begin{itemize}
  \item Developed a library for linear algebra running on ARM Cortex-M MCU
\end{itemize}

\noindent Reason for changing job:

\begin{itemize}
  \item Due to commissioned embedded AI project frozen
\end{itemize}

%%%%%%%%%%%%%%%%
\subsection*{February 2018 - July 2018: Software Architect at (contract employee) SHINKAWA LTD.}

\noindent Responsibilities:

\begin{itemize}
  \item Researched and developed new software platform for wire bonding during semiconductor device fabrication
\end{itemize}

\noindent Key Achievements:

\begin{itemize}
  \item Created a parser to understand SHINKAWA own embedded script language
  \item Evaluated EtherCAT\footnote{\href{https://www.ethercat.org/}{\tt https://www.ethercat.org/}} protocol for the realtime application
\end{itemize}

\noindent Reason for changing job:

\begin{itemize}
  \item Initially, I joined SHINKAWA as a "Linux expert," but as the team was organized, the person they needed was a hardware engineer
\end{itemize}

%%%%%%%%%%%%%%%%
\subsection*{August 2014 - October 2017: Part-time Researcher at RIKEN Advanced Institute for Computational Science}

\noindent Responsibilities:

\begin{itemize}
  \item Researched functional programming for embedded platform
\end{itemize}

\noindent Key Achievements:

\begin{itemize}
  \item Published some research papers\footnote{\href{http://www.metasepi.org/papers.html}{\tt http://www.metasepi.org/papers.html}}
\end{itemize}

\noindent Reason for changing job:

\begin{itemize}
  \item Because RIKEN AICS was going to outsource compilers for supercomputers instead of making them, and my research theme was no longer in line with theirs
\end{itemize}

%%%%%%%%%%%%%%%%
\subsection*{November 2016 - October 2017: Expert Engineer (permanent employee) at SELTECH CORPORATION}

\noindent Responsibilities:

\begin{itemize}
  \item Researched and developed new Secure-OS for ARM Cortex-M platform
\end{itemize}

\noindent Reason for changing job:

\begin{itemize}
  \item Due to the dissolution of the Secure OS project for Cortex-M microcontrollers that I was working on
\end{itemize}

%%%%%%%%%%%%%%%%
\subsection*{February 2016 - November 2016: Software Engineer (contract employee) at Life Robotics Inc.}

\noindent Responsibilities:

\begin{itemize}
  \item Developed GUI application running on Ubuntu OS, using C++ and Qt\footnote{\href{https://www.qt.io/}{\tt https://www.qt.io/}} for single arm robot
\end{itemize}

\noindent Key Achievements:

\begin{itemize}
  \item Designed a network protocol for the robotics application
\end{itemize}

\noindent Reason for changing job:

\begin{itemize}
  \item Commuting time was taking 4 hours round trip, and I was reaching my physical limit
\end{itemize}

%%%%%%%%%%%%%%%%
\subsection*{March 2015 - February 2016: System Enginner (contract employee) at Centillion Japan Co., Ltd.}

\noindent Responsibilities:

\begin{itemize}
  \item Technical support for stock chart application using JavaScript
  \item Maintained MySQL database servers
  \item Manager for offshore development in China
\end{itemize}

\noindent Key Achievements:

\begin{itemize}
  \item Launched new IoT business for farming
  \item Design a platform\footnote{\href{https://github.com/centillion-tech/kick-r}{\tt https://github.com/centillion-tech/kick-r}} to accelerate R\footnote{\href{https://www.r-project.org/}{\tt https://www.r-project.org/}} programs
\end{itemize}

\noindent Reason for changing job:

\begin{itemize}
  \item For the opportunity to return to embedded development after a career that had once moved to the web server side
\end{itemize}

%%%%%%%%%%%%%%%%
\subsection*{September 2014 - December 2014: Software engineer (trustee agreement) at Axsh co., LTD.}

\noindent Responsibilities:

\begin{itemize}
  \item Developed an OpenFlow application named ``OpenVNet''\footnote{\href{https://github.com/axsh/openvnet}{\tt https://github.com/axsh/openvnet}}
\end{itemize}

\noindent Key Achievements:

\begin{itemize}
  \item Provisioned and automated deploying the OpenVNet on AWS platform using Ruby and GNU make
\end{itemize}

\noindent Reason for changing job:

\begin{itemize}
  \item Because they did not receive commensurate compensation for the hours worked
\end{itemize}

%%%%%%%%%%%%%%%%
\subsection*{March 2012 - July 2013: Software Engineer (permanent employee) at MIRACLE LINUX CORPORATION}

\noindent Responsibilities:

\begin{itemize}
  \item Developed own Digital Signage platform running on Intel architecture using Linux OS, C++, OpenGL, GTK+\footnote{\href{https://www.gtk.org/}{\tt https://www.gtk.org/}}, GStreamer\footnote{\href{https://gstreamer.freedesktop.org/}{\tt https://gstreamer.freedesktop.org/}}
  \item Supported and debugged own Linux distribution
\end{itemize}

\noindent Key Achievements:

\begin{itemize}
  \item Debugged and fixed a race condition in the SMP kernel on PowerPC platform
  \item Debugged and fixed bug of crash\footnote{\href{http://people.redhat.com/{\textasciitilde}anderson/}{\tt http://people.redhat.com/{\textasciitilde}anderson/}} command's PowerPC virtual memory
  \item Designed new Windows installer using NSIS\footnote{\href{http://nsis.sourceforge.net/}{\tt http://nsis.sourceforge.net/}}
\end{itemize}

\noindent Reason for changing job:

\begin{itemize}
  \item I wanted to do business with a Haskell compiler that I had been developing privately
\end{itemize}

%%%%%%%%%%%%%%%%
\subsection*{April 2001 - February 2012: Software Development Engineer (permanent employee) at Ricoh Company, Ltd.}

\noindent Responsibilities:

\begin{itemize}
  \item Developed own platform for multi-function printer based on NetBSD OS
\end{itemize}

\noindent Key Achievements:

\begin{itemize}
  \item Developed OptionBIOS and bootloader for the platform on Intel architecture
  \item Designed secure boot for the platform on Intel architecture
  \item Compressed boot time of the printer onto 10 seconds
  \item Verified m:n POSIX thread library
\end{itemize}

\noindent Reason for changing job:

\begin{itemize}
  \item Because we wanted to develop a smaller organization with faster decision making
\end{itemize}

\section*{Education}

\begin{itemize}
  \item March 2001: Master of Engineering from Department of Electrical and Electronic Engineering, Tokyo Metropolitan University. \\
    The thesis: ``Multimode Quartz Crystal Microbalance''\footnote{\href{http://ci.nii.ac.jp/naid/110004076869}{\tt http://ci.nii.ac.jp/naid/110004076869}}
\end{itemize}

\section*{Publications and Reports}

\begin{itemize}
  \item Kiwamu Okabe and Hongwei Xi. ``Arduino programing of ML-style in ATS''\footnote{\href{http://www.metasepi.org/doc/metasepi-icfp2015-arduino-ats.pdf}{\tt http://www.metasepi.org/doc/metasepi-icfp2015-arduino-ats.pdf}}. ML workshop, 2015.
  \item Kiwamu Okabe and Takayuki Muranushi. ``Systems Demonstration: Writing NetBSD Sound Drivers in Haskell''\footnote{\href{http://metasepi.org/doc/metasepi-icfp2014-demo.pdf}{\tt http://metasepi.org/doc/metasepi-icfp2014-demo.pdf}}. Haskell Symposium, 2014.
  \item Kiwamu Okabe. ``ATS言語を使って不変条件をAPIに強制する''.\footnote{\href{http://www.metasepi.org/doc/20141101\_prosym\_summer2014.pdf}{\tt http://www.metasepi.org/doc/20141101\_prosym\_summer2014.pdf}} 夏のプログラミング・シンポジウム 2014, 2014.
  \item Kiwamu Okabe, Hiroki MIZUNO and Hidekazu SEGAWA. ``強い型によるOSの開発手法の提案''\footnote{\href{http://metasepi.org/doc/20140110\_prosym55.pdf}{\tt http://metasepi.org/doc/20140110\_prosym55.pdf}}. 第55回プログラミング・シンポジウム, 2014.
\end{itemize}

\section*{Activities}

\subsection*{Open-source projects}

\subsubsection*{Metasepi Project\footnote{\href{http://metasepi.org/}{\tt http://metasepi.org/}}}
\begin{itemize}
\item Challenge to create an open-source Unix-like operating system designed with strong type such as ML or Haskell.
\item Rewriting NetBSD kernel using Ajhc Haskell compiler. \href{https://github.com/metasepi/netbsd-arafura-s1}{\tt https://github.com/metasepi/netbsd-arafura-s1}
\end{itemize}

\subsubsection*{Ajhc Haskell compiler\footnote{\href{http://ajhc.metasepi.org/}{\tt http://ajhc.metasepi.org/}}}
\begin{itemize}
\item Extend and add embedded features to Jhc Haskell Compiler \href{http://repetae.net/computer/jhc/}{\tt http://repetae.net/computer/jhc/}.
\item Ajhc has thread-safe and reentrant runtime. Also has Erlang style GC. It means Ajhc's Haskell context has own GC heap. GC can run on tiny CPU such as Cortex-M3 with 32kB RAM.
\end{itemize}

\subsubsection*{Japan ATS User Group\footnote{\href{http://jats-ug.metasepi.org/}{\tt http://jats-ug.metasepi.org/}}}
\begin{itemize}
\item An user group for ATS language promotion of utilization. Translating ATS documents into Japanese.
\end{itemize}

\subsubsection*{Debian Maintainer\footnote{\href{http://qa.debian.org/developer.php?login=kiwamu@debian.or.jp}{\tt http://qa.debian.org/developer.php?login=kiwamu@debian.or.jp}}}
\begin{itemize}
\item Maintained uim package at Debian squeeze, and packages using Haskell at sid.
\end{itemize}

\bigskip
\footer

\end{document}
