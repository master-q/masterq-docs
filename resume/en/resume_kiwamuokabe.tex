% LaTeX Curriculum Vitae Template
%
% Copyright (C) 2004-2009 Jason Blevins <jrblevin@sdf.lonestar.org>
% http://jblevins.org/projects/cv-template/
%
% You may use use this document as a template to create your own CV
% and you may redistribute the source code freely. No attribution is
% required in any resulting documents. I do ask that you please leave
% this notice and the above URL in the source code if you choose to
% redistribute this file.

\documentclass[letterpaper]{article}

\usepackage[dvipdfmx]{graphicx}
\usepackage{fancybox}
\usepackage{longtable}
\usepackage{fancyvrb}
\usepackage[dvipdfmx]{hyperref}
\usepackage{url}
\usepackage[dvipdfmx]{color}
\usepackage{nextpage}
\usepackage{float}
\usepackage[table,dvipdfmx]{xcolor}
\usepackage{wrapfig}
\usepackage{tabularx}
\usepackage{multicol}
\usepackage{makeidx}
\usepackage{ascmac} % screen
\usepackage[dvips]{xy} % for advi workaround. Bug #452044

\usepackage{geometry}

% Comment the following lines to use the default Computer Modern font
% instead of the Palatino font provided by the mathpazo package.
% Remove the 'osf' bit if you don't like the old style figures.
%\usepackage[T1]{fontenc}
%\usepackage[sc,osf]{mathpazo}

% Set your name here
\def\name{Kiwamu Okabe - Research and Development Engineer}

% Replace this with a link to your CV if you like, or set it empty
% (as in \def\footerlink{}) to remove the link in the footer:
\def\footer{
  \begin{center}
    \begin{footnotesize}
      Last updated: \today
    \end{footnotesize}
  \end{center}
}

% The following metadata will show up in the PDF properties
\hypersetup{
  colorlinks = true,
  urlcolor = black,
  pdfauthor = {\name},
  pdfkeywords = {economics, statistics, mathematics},
  pdftitle = {\name: Curriculum Vitae},
  pdfsubject = {Curriculum Vitae},
  pdfpagemode = UseNone
}

\geometry{
  body={6.5in, 8.5in},
  left=1.0in,
  top=1.25in
}

% Customize page headers
\pagestyle{myheadings}
\markright{\name}
\thispagestyle{empty}

% Custom section fonts
\usepackage{sectsty}
\sectionfont{\rmfamily\mdseries\Large}
\subsectionfont{\rmfamily\mdseries\itshape\large}

% Other possible font commands include:
% \ttfamily for teletype,
% \sffamily for sans serif,
% \bfseries for bold,
% \scshape for small caps,
% \normalsize, \large, \Large, \LARGE sizes.

% Don't indent paragraphs.
\setlength\parindent{0.2em}

%% Make lists without bullets
%\renewenvironment{itemize}{
%  \begin{list}{}{
%    \setlength{\leftmargin}{1.5em}
%  }
%}{
%  \end{list}
%}

\begin{document}

% Place name at left
{\huge \name}

% Alternatively, print name centered and bold:
%\centerline{\huge \bf \name}

\vspace{0.25in}

\begin{minipage}{0.3\linewidth}
  \begin{tabular}{ll}
    Phone: & +81-90-3524-7064 \\
    Email: & \href{mailto:kiwamu@debian.or.jp}{\tt kiwamu@debian.or.jp} \\
    Homepage: & \href{http://www.masterq.net/}{\tt http://www.masterq.net/} \\
  \end{tabular}
\end{minipage}

\section*{Interests}

ATS programming, Kernel programming, Debian project, Strongly typed language, Haskell programming.

% xxx 強みアピールをもっと
% Interests の項目も、言語とOSの列挙になっているので、lowlevelなレイヤー(NetBSD kernel programming)、ハイレベルな抽象化(Haskell/ATS)、そしてそらをつなぐ未来(metasepi)、とか何とかかいとくとどのような理由/ストーリーでこれらに興味があるのかが見えてくるかと
% 経験 => 問題点の認識 => 解決策の模索 => 要素技術探索している今
% たぶん文面上はmetasepiではなく広い文脈で「低レイヤーと抽象レイヤーをつなげたい」くらいにしとくと可能性が広がってよいのでは

\section*{Work Experience}
\subsection*{March 2015 - Present: System Enginner at Centillion Japan Co., Ltd.}

\begin{itemize}
  \item Software engineering for cloud computing
\end{itemize}

\subsection*{August 2014 - Present: Part-time researcher at RIKEN Advanced Institute for Computational Science}

\begin{itemize}
  \item Research embedded functional programming
\end{itemize}

\subsection*{February 2014 - Present: Self-employed Software Engineer at METASEPI DESIGN}

\begin{itemize}
  \item Support to develop any embedded software
  \item ATS language consulting
  \item Manage Metasepi Project and develop the core technology
\end{itemize}

\subsection*{September 2014 - December 2014: Software engineer at Axsh co., LTD.}

\begin{itemize}
  \item Develop an OpenFlow application named ``OpenVNet'' \footnote{\href{https://github.com/axsh/openvnet}{\tt https://github.com/axsh/openvnet}}
\end{itemize}

\subsection*{July 2013 - February 2014: Freelance}

\begin{itemize}
  \item Research and develop Ajhc Haskell Compiler
\end{itemize}

%%%%%%%%%%%%%%%%
\subsection*{March 2012 - July 2013: MIRACLE LINUX CORPORATION}

\begin{itemize}
  \item Verify PowerPC Linux and debug/fix SMP Race Condition
  \item Debug and fix PowerPC crash command's virtual memory paging BUG
  \item Design new Windows installer using NSIS
  \item Introduce and maintain new git server for internal use
  \item Verify and tune performance of Digital Signage on new hardware
\end{itemize}

%%%%%%%%%%%%%%%%
\subsection*{April 2001 - February 2012: Software Development Engineer at Ricoh Company, Ltd.}
\subsubsection*{April 2010: Port OS to new x86 hardware}
I ported NetBSD-2.0 to new x86 hardware, and calculated boot time before porting OS.

\subsubsection*{April 2008: Develop and technical support NetBSD OS}
I became technical leader to maintain and support Ricoh's multifunction printer.
Also I debuged and fix many kernel level BUGs.
I designed power down process by software trigger.

\subsubsection*{April 2006: Develop POSIX thread library}
I replaced user level POSIX thread library with NetBSD-2.0 m:n thread.
My efforts are following.
\begin{itemize}
  \item Research the presence of thread-safe and cancel-safe in kernel, libc and libpthread.
  \item Write test code to get thread-safe and cancel-safe.
  \item Add API into the m:n thread library. Because printer application depend on the missing API.
  \item Debug and fix signal BUGs in the m:n thread library.
\end{itemize}

\subsubsection*{June 2004: Tune multifunction printer boot time as 10 seconds}
Before my tune, the printer boot time is 30 seconds.
Ricoh's multifunction printer ``imagio''
\footnote{\href{http://www.ricoh.co.jp/imagio/}{\tt http://www.ricoh.co.jp/imagio/}}
use the method and tool, ever now.
Develop a tool analyze log dumped by bootloader, kernel, init process and application.
Analyze IPC network, and advice how to get speed to application developer.
Split application as groups for pre-boot and post-boot.

\subsubsection*{October 2003: Develop new BIOS for multifunction printer}
Develop new custom BIOS together at the United States.

\subsubsection*{November 2002: Design secure boot for multifunction printer on x86}
I re-designed bootloader to support public key algorithm authentication,
and designed the format include secure key in SD card

\subsubsection*{July 2001: Develop BIOS and bootloader for multifunction printer on x86 architecture}
My first work at Ricoh. Ricoh's multifunction printer
``imagio''\footnote{\href{http://www.ricoh.co.jp/imagio/}{\tt http://www.ricoh.co.jp/imagio/}}
use my bootloader design, ever now.
Also I designed hardware dependent data structure to configure bootloader and kernel,
and Option BIOS to boot on SD card

\section*{Education}

\begin{itemize}
  \item March 2001: Master of Engineering from Department of Electrical and Electronic Engineering, Tokyo Metropolitan University. \\
    The thesis: ``Multimode Quartz Crystal Microbalance''
    \footnote{\href{http://ci.nii.ac.jp/naid/110004076869}{\tt http://ci.nii.ac.jp/naid/110004076869}}
\end{itemize}

\section*{Publications and Reports}

\begin{itemize}
  \item Kiwamu Okabe and Hongwei Xi. ``Arduino programing of ML-style in ATS'' \footnote{\href{http://www.metasepi.org/doc/metasepi-icfp2015-arduino-ats.pdf}{\tt http://www.metasepi.org/doc/metasepi-icfp2015-arduino-ats.pdf}}. ML workshop, 2015.
  \item Kiwamu Okabe and Takayuki Muranushi. ``Systems Demonstration: Writing NetBSD Sound Drivers in Haskell'' \footnote{\href{http://metasepi.org/doc/metasepi-icfp2014-demo.pdf}{\tt http://metasepi.org/doc/metasepi-icfp2014-demo.pdf}}. Haskell Symposium, 2014.
  \item Kiwamu Okabe, Hiroki MIZUNO and Hidekazu SEGAWA. ``強い型によるOSの開発手法の提案'' \footnote{\href{http://metasepi.org/doc/20140110\_prosym55.pdf}{\tt http://metasepi.org/doc/20140110\_prosym55.pdf}}. 第55回プログラミング・シンポジウム, 2014.
  \item Kiwamu Okabe. ``ATS言語を使って不変条件をAPIに強制する''. \footnote{\href{http://www.metasepi.org/doc/20141101\_prosym\_summer2014.pdf}{\tt http://www.metasepi.org/doc/20141101\_prosym\_summer2014.pdf}} 夏のプログラミング・シンポジウム 2014, 2014.
\end{itemize}

\section*{Activities}

\subsection*{Open-source projects}

\subsubsection*{Metasepi Project \footnote{\href{http://metasepi.org/}{\tt http://metasepi.org/}}}
\begin{itemize}
\item Challenge to create an open-source Unix-like operating system designed with strong type such as ML or Haskell.
\item Rewriting NetBSD kernel using Ajhc Haskell compiler. \href{https://github.com/metasepi/netbsd-arafura-s1}{\tt https://github.com/metasepi/netbsd-arafura-s1}
\end{itemize}

\subsubsection*{Ajhc Haskell compiler \footnote{\href{http://ajhc.metasepi.org/}{\tt http://ajhc.metasepi.org/}}}
\begin{itemize}
\item Extend and add embedded features to Jhc Haskell Compiler \href{http://repetae.net/computer/jhc/}{\tt http://repetae.net/computer/jhc/}.
\item Ajhc has thread-safe and reentrant runtime. Also has Erlang style GC. It means Ajhc's Haskell context has own GC heap. GC can run on tiny CPU such as Cortex-M3 with 32kB RAM.
\end{itemize}

\subsubsection*{Japan ATS User Group \footnote{\href{http://jats-ug.metasepi.org/}{\tt http://jats-ug.metasepi.org/}}}
\begin{itemize}
\item An user group for ATS language \href{http://www.ats-lang.org/}{\tt http://www.ats-lang.org/} promotion of utilization. Translating ATS documents into Japanese.
\end{itemize}

\subsubsection*{Debian Maintainer \footnote{\href{http://qa.debian.org/developer.php?login=kiwamu@debian.or.jp}{\tt http://qa.debian.org/developer.php?login=kiwamu@debian.or.jp}}}
\begin{itemize}
\item Maintained uim package at Debian squeeze, and packages using Haskell at sid.
\end{itemize}

\subsubsection*{Carettah \footnote{\href{http://carettah.masterq.net/}{\tt http://carettah.masterq.net/}}}
\begin{itemize}
\item A presentation tool written with Haskell. My slides \href{http://www.slideshare.net/master\_q/}{\tt http://www.slideshare.net/master\_q/} are created by the tool.
\end{itemize}

\section*{Computer Skills}

\begin{itemize}
  \item Languages: Haskell, C, ATS, Intel assembler, Ruby
  \item Platforms: Linux, NetBSD, Android NDK, MinGW
\end{itemize}

\section*{Reference available upon request}

\begin{itemize}
  \item Kentaro Kuroiwa Research Chief - Centillion Japan Co., Ltd.
  \item Yasuhiro Yamazaki CEO - Axsh Co., Ltd.
  \item Takayuki Muranushi - RIKEN Advanced Institute for Computational Science
  \item Takashi KODAMA CEO - MIRACLE LINUX CORPORATION
  \item Shigeya SENDA - Ricoh Company, Ltd.
  \item Hitoshi Sekimoto Professor - Tokyo Metropolitan University, Department of Electrical and Electronic Engineering
\end{itemize}

\bigskip
\footer

\end{document}
