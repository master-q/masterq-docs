% LaTeX Curriculum Vitae Template
%
% Copyright (C) 2004-2009 Jason Blevins <jrblevin@sdf.lonestar.org>
% http://jblevins.org/projects/cv-template/
%
% You may use use this document as a template to create your own CV
% and you may redistribute the source code freely. No attribution is
% required in any resulting documents. I do ask that you please leave
% this notice and the above URL in the source code if you choose to
% redistribute this file.

\documentclass[letterpaper]{article}

\usepackage[dvipdfmx]{graphicx}
\usepackage{fancybox}
\usepackage{longtable}
\usepackage{fancyvrb}
\usepackage[dvipdfmx]{hyperref}
\usepackage{url}
\usepackage[dvipdfmx]{color}
\usepackage{nextpage}
\usepackage{float}
\usepackage{wrapfig}
\usepackage{tabularx}
\usepackage{multicol}
\usepackage{makeidx}
\usepackage{ascmac} % screen
\usepackage[dvips]{xy} % for advi workaround. Bug #452044

\usepackage{geometry}

% Comment the following lines to use the default Computer Modern font
% instead of the Palatino font provided by the mathpazo package.
% Remove the 'osf' bit if you don't like the old style figures.
%\usepackage[T1]{fontenc}
%\usepackage[sc,osf]{mathpazo}

% Set your name here
\def\name{Kiwamu Okabe}

% Replace this with a link to your CV if you like, or set it empty
% (as in \def\footerlink{}) to remove the link in the footer:
\def\footer{
  \begin{center}
    \begin{footnotesize}
      Last updated: \today
    \end{footnotesize}
  \end{center}
}

% The following metadata will show up in the PDF properties
\hypersetup{
  colorlinks = true,
  urlcolor = black,
  pdfauthor = {\name},
  pdfkeywords = {economics, statistics, mathematics},
  pdftitle = {\name: Curriculum Vitae},
  pdfsubject = {Curriculum Vitae},
  pdfpagemode = UseNone
}

\geometry{
  body={6.5in, 8.5in},
  left=1.0in,
  top=1.25in
}

% Customize page headers
\pagestyle{myheadings}
\markright{\name}
\thispagestyle{empty}

% Custom section fonts
\usepackage{sectsty}
\sectionfont{\rmfamily\mdseries\Large}
\subsectionfont{\rmfamily\mdseries\itshape\large}

% Other possible font commands include:
% \ttfamily for teletype,
% \sffamily for sans serif,
% \bfseries for bold,
% \scshape for small caps,
% \normalsize, \large, \Large, \LARGE sizes.

% Don't indent paragraphs.
\setlength\parindent{0.2em}

%% Make lists without bullets
%\renewenvironment{itemize}{
%  \begin{list}{}{
%    \setlength{\leftmargin}{1.5em}
%  }
%}{
%  \end{list}
%}

\begin{document}

% Place name at left
{\huge \name}

% Alternatively, print name centered and bold:
%\centerline{\huge \bf \name}

\vspace{0.25in}

\begin{minipage}{0.3\linewidth}
  \begin{tabular}{ll}
    Phone: & +81-90-3524-7064 \\
    Email: & \href{mailto:kiwamu@debian.or.jp}{\tt kiwamu@debian.or.jp} \\
    Homepage: & \href{http://masterq.metasepi-design.com/}{\tt http://masterq.metasepi-design.com/} \\
  \end{tabular}
\end{minipage}

\section*{Brief}

I launched my career on developing embedded devices using Unix-like kernel at Ricoh Company, Ltd. The experience was not only for designing device drivers but also including debug of virtual memory. And I have experience in IoT platfom such as ARM Cortex-M MCU and RTOS. Also I learned application design using functional language such as Haskell, and published some research papers\footnote{\href{http://www.metasepi.org/papers.html}{\tt http://www.metasepi.org/papers.html}} about such languages.

\section*{Skill Set}

\subsection*{Deep knowledge for Unix-like Kernel and User Space}

I am an expert for Unix-like kernel such as Linux, because I provided technical support for NetBSD, which is a Unix-like OS similar to Linux, at Ricoh. My skill is not only for NetBSD but also Linux. In fact, a race condition bug in PowerPC Linux kernel was fixed by me in only five days at MIRACLE LINUX.

\subsection*{Wide experience in IoT platform}

I have a wide experience in IoT platform such as FreeRTOS, ChibiOS/RT\footnote{\href{http://www.chibios.org/}{\tt http://www.chibios.org/}}, ARM Cortex-M, ESP8266, AVR and MSP430. Also I launched a new IoT business using TWELITE wireless platform\footnote{\href{https://mono-wireless.com/}{\tt https://mono-wireless.com/}} at Centillion Japan.

\subsection*{Leadership}

I was leading a technical team of twenty people to support the OS at Ricoh. And also I have experience in leading offshore team in china to maintain web application at Centillion Japan.

\subsection*{Research Security and Quality}

A prototype of own Secure-OS similar to OP-TEE\footnote{\href{https://github.com/OP-TEE/optee\_os}{\tt https://github.com/OP-TEE/optee\_os}} was designed by me at SELTECH. It runs with the other RTOS on ARM Cortex-M MCU. Also I have a wealth of experience in strong static typing language such as Haskell and verification of C language such as VeriFast\footnote{\href{https://github.com/verifast/verifast}{\tt https://github.com/verifast/verifast}}, which are useful to keep the quality of products.

\newpage

\section*{Work Experience}

%%%%%%%%%%%%%%%%
\subsection*{July 2013 - Present: Freelance Researcher}

Responsibilities:

\begin{itemize}
  \item Researching and developing Ajhc Haskell Compiler\footnote{\href{http://ajhc.metasepi.org/}{\tt http://ajhc.metasepi.org/}}
  \item ATS language\footnote{\href{http://www.ats-lang.org/}{\tt http://www.ats-lang.org/}} evangelist for embedded devices
  \item Verification evangelist using VeriFast\footnote{\href{https://github.com/verifast/verifast}{\tt https://github.com/verifast/verifast}}, which is a verifier C language programs annotated with preconditions and postconditions
\end{itemize}

Key Achievements:

\begin{itemize}
  \item Published some research papers\footnote{\href{http://www.metasepi.org/papers.html}{\tt http://www.metasepi.org/papers.html}}
  \item Translated ATS documents\footnote{\href{http://jats-ug.metasepi.org/}{\tt http://jats-ug.metasepi.org/}} into Japanese
  \item Translated VeriFast documents\footnote{\href{https://github.com/jverifast-ug/translate}{\tt https://github.com/jverifast-ug/translate}} into Japanese
\end{itemize}

%%%%%%%%%%%%%%%%
\subsection*{February 2018 - July 2018: Software Architect at SHINKAWA LTD.}

Responsibilities:

\begin{itemize}
  \item Researched and developed new software platform for wire bonding during semiconductor device fabrication
\end{itemize}

Key Achievements:

\begin{itemize}
  \item Created a parser to understand SHINKAWA own embedded script language
  \item Evaluated EtherCAT\footnote{\href{https://www.ethercat.org/}{\tt https://www.ethercat.org/}} protocol for the realtime application
\end{itemize}

%%%%%%%%%%%%%%%%
\subsection*{August 2014 - October 2017: Part-time Researcher at RIKEN Advanced Institute for Computational Science}

Responsibilities:

\begin{itemize}
  \item Researched functional programming for embedded platform
\end{itemize}

Key Achievements:

\begin{itemize}
  \item Published some research papers\footnote{\href{http://www.metasepi.org/papers.html}{\tt http://www.metasepi.org/papers.html}}
\end{itemize}

%%%%%%%%%%%%%%%%
\subsection*{November 2016 - October 2017: Expert Engineer (permanent employee) at SELTECH CORPORATION}

Responsibilities:

\begin{itemize}
  \item Researched and developed new Secure-OS for ARM Cortex-M platform
\end{itemize}

%%%%%%%%%%%%%%%%
\subsection*{February 2016 - November 2016: Software Engineer (contract employee) at Life Robotics Inc.}

Responsibilities:

\begin{itemize}
  \item Developed GUI application running on Ubuntu OS, using C++ and Qt\footnote{\href{https://www.qt.io/}{\tt https://www.qt.io/}} for single arm robot
\end{itemize}

Key Achievements:

\begin{itemize}
  \item Designed a network protocol for the robotics application
\end{itemize}

%%%%%%%%%%%%%%%%
\subsection*{March 2015 - February 2016: System Enginner (contract employee) at Centillion Japan Co., Ltd.}

Responsibilities:

\begin{itemize}
  \item Technical support for stock chart application using JavaScript
  \item Maintained MySQL database servers
  \item Manager for offshore development in China
\end{itemize}

Key Achievements:

\begin{itemize}
  \item Launched new IoT business for farming
  \item Design a platform\footnote{\href{https://github.com/centillion-tech/kick-r}{\tt https://github.com/centillion-tech/kick-r}} to accelerate R programs
\end{itemize}

%%%%%%%%%%%%%%%%
\subsection*{September 2014 - December 2014: Software engineer (trustee agreement) at Axsh co., LTD.}

Responsibilities:

\begin{itemize}
  \item Developed an OpenFlow application named ``OpenVNet''\footnote{\href{https://github.com/axsh/openvnet}{\tt https://github.com/axsh/openvnet}}
\end{itemize}

Key Achievements:

\begin{itemize}
  \item Provisioned and automated deploying the OpenVNet on AWS platform using Ruby and GNU make
\end{itemize}

%%%%%%%%%%%%%%%%
\subsection*{March 2012 - July 2013: Software Engineer (permanent employee) at MIRACLE LINUX CORPORATION}

Responsibilities:

\begin{itemize}
  \item Developed own Digital Signage platform running on Intel architecture using Linux OS, C++, OpenGL, GTK+ and GStreamer
  \item Supported and debugged own Linux distribution
\end{itemize}

Key Achievements:

\begin{itemize}
  \item Debugged and fixed a race condition in the SMP kernel on PowerPC platform
  \item Debugged and fixed bug of crash\footnote{\href{http://people.redhat.com/{\textasciitilde}anderson/}{\tt http://people.redhat.com/{\textasciitilde}anderson/}} command's PowerPC virtual memory
  \item Designed new Windows installer using NSIS\footnote{\href{http://nsis.sourceforge.net/}{\tt http://nsis.sourceforge.net/}}
\end{itemize}

%%%%%%%%%%%%%%%%
\subsection*{April 2001 - February 2012: Software Development Engineer (permanent employee) at Ricoh Company, Ltd.}

Responsibilities:

\begin{itemize}
  \item Developed own platform for multi-function printer based on NetBSD OS
\end{itemize}

Key Achievements:

\begin{itemize}
  \item Developed OptionBIOS and bootloader for the platform on Intel architecture
  \item Designed secure boot for the platform on Intel architecture
  \item Compressed boot time of the printer onto 10 seconds
  \item Verified m:n POSIX thread library
\end{itemize}

\section*{Education}

\begin{itemize}
  \item March 2001: Master of Engineering from Department of Electrical and Electronic Engineering, Tokyo Metropolitan University. \\
    The thesis: ``Multimode Quartz Crystal Microbalance''\footnote{\href{http://ci.nii.ac.jp/naid/110004076869}{\tt http://ci.nii.ac.jp/naid/110004076869}}
\end{itemize}

\section*{Publications and Reports}

\begin{itemize}
  \item Kiwamu Okabe and Hongwei Xi. ``Arduino programing of ML-style in ATS''\footnote{\href{http://www.metasepi.org/doc/metasepi-icfp2015-arduino-ats.pdf}{\tt http://www.metasepi.org/doc/metasepi-icfp2015-arduino-ats.pdf}}. ML workshop, 2015.
  \item Kiwamu Okabe and Takayuki Muranushi. ``Systems Demonstration: Writing NetBSD Sound Drivers in Haskell''\footnote{\href{http://metasepi.org/doc/metasepi-icfp2014-demo.pdf}{\tt http://metasepi.org/doc/metasepi-icfp2014-demo.pdf}}. Haskell Symposium, 2014.
  \item Kiwamu Okabe. ``ATS言語を使って不変条件をAPIに強制する''.\footnote{\href{http://www.metasepi.org/doc/20141101\_prosym\_summer2014.pdf}{\tt http://www.metasepi.org/doc/20141101\_prosym\_summer2014.pdf}} 夏のプログラミング・シンポジウム 2014, 2014.
  \item Kiwamu Okabe, Hiroki MIZUNO and Hidekazu SEGAWA. ``強い型によるOSの開発手法の提案''\footnote{\href{http://metasepi.org/doc/20140110\_prosym55.pdf}{\tt http://metasepi.org/doc/20140110\_prosym55.pdf}}. 第55回プログラミング・シンポジウム, 2014.
\end{itemize}

\section*{Activities}

\subsection*{Open-source projects}

\subsubsection*{Metasepi Project\footnote{\href{http://metasepi.org/}{\tt http://metasepi.org/}}}
\begin{itemize}
\item Challenge to create an open-source Unix-like operating system designed with strong type such as ML or Haskell.
\item Rewriting NetBSD kernel using Ajhc Haskell compiler. \href{https://github.com/metasepi/netbsd-arafura-s1}{\tt https://github.com/metasepi/netbsd-arafura-s1}
\end{itemize}

\subsubsection*{Ajhc Haskell compiler\footnote{\href{http://ajhc.metasepi.org/}{\tt http://ajhc.metasepi.org/}}}
\begin{itemize}
\item Extend and add embedded features to Jhc Haskell Compiler \href{http://repetae.net/computer/jhc/}{\tt http://repetae.net/computer/jhc/}.
\item Ajhc has thread-safe and reentrant runtime. Also has Erlang style GC. It means Ajhc's Haskell context has own GC heap. GC can run on tiny CPU such as Cortex-M3 with 32kB RAM.
\end{itemize}

\subsubsection*{Japan ATS User Group\footnote{\href{http://jats-ug.metasepi.org/}{\tt http://jats-ug.metasepi.org/}}}
\begin{itemize}
\item An user group for ATS language promotion of utilization. Translating ATS documents into Japanese.
\end{itemize}

\subsubsection*{Debian Maintainer\footnote{\href{http://qa.debian.org/developer.php?login=kiwamu@debian.or.jp}{\tt http://qa.debian.org/developer.php?login=kiwamu@debian.or.jp}}}
\begin{itemize}
\item Maintained uim package at Debian squeeze, and packages using Haskell at sid.
\end{itemize}

\section*{Computer Skills}

\begin{itemize}
  \item Languages: C, C++, Haskell, Intel/ARM assembler, Ruby, OCaml, Python, Erlang, JavaScript, R
  \item Platforms: Linux, NetBSD, FreeRTOS, ChibiOS/RT, Android NDK, Cygwin, MinGW, Bare metal
\end{itemize}

\section*{Reference available upon request}

\begin{itemize}
  \item Hiroshi Munakata CTO - SHINKAWA LTD.
  \item Shoi Egawa CEO - SELTECH CORPORATION
  \item Woo-Keun Yoon CEO - Life Robotics Inc.
  \item Kentaro Kuroiwa Research Chief - Centillion Japan Co., Ltd.
  \item Yasuhiro Yamazaki CEO - Axsh Co., Ltd.
  \item Junichiro Makino Team Leader - RIKEN Advanced Institute for Computational Science
  \item Takashi KODAMA CEO - MIRACLE LINUX CORPORATION
  \item Shigeya SENDA - Ricoh Company, Ltd.
  \item Hitoshi Sekimoto Professor - Tokyo Metropolitan University, Department of Electrical and Electronic Engineering
\end{itemize}

\bigskip
\footer

\end{document}
