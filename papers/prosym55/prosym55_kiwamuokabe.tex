%%
%% 研究報告用スイッチ
%% [techrep]
%%
%% 欧文表記無しのスイッチ(etitle,jkeyword,eabstract,ekeywordは任意)
%% [noauthor]
%%

%\documentclass[submit,techreq]{ipsj}
\documentclass[submit,techreq,noauthor]{ipsj}


\usepackage[dvips]{graphicx}
\usepackage{latexsym}

\def\Underline{\setbox0\hbox\bgroup\let\\\endUnderline}
\def\endUnderline{\vphantom{y}\egroup\smash{\underline{\box0}}\\}
\def\|{\verb|}


\begin{document}

% Title %%%%%%%%%%%%%%%%%%%%%%%%%%%%%%%%%
\title{強い型によるOSの開発手法の提案}

\affiliate{METASEPI}{Metasepi Project}
\affiliate{OCAML-NAGOYA}{ocaml-nagoya}

\author{岡部 究}{Kiwamu Okabe}{METASEPI}[kiwamu@debian.or.jp]
\author{水野 洋樹}{Hiroki MIZUNO}{OCAML-NAGOYA}[mzp@ocaml.jp]
\author{瀬川 秀一}{Hidekazu SEGAWA}{}[]

\begin{abstract}
実用化されているOSは現在でもC言語によって設計されている.
一方アプリケーション領域ではMLやHaskellのような型推論を持つ言語を用いて
安全な設計をする手法が確立されている.
本稿では強い型を使った安全な設計手法をOSのような低レベルのソフトウェアに適用する方法を提案する.
\end{abstract}

\begin{jkeyword}
プログラミング・シンポジウム,プログラミング言語,コンパイラ,Haskell,OS
\end{jkeyword}

\maketitle

% Body %%%%%%%%%%%%%%%%%%%%%%%%%%%%%%%%%
\section{はじめに}

筆者らは既存のjhcコンパイラ \cite{jhc}
に組込み向け拡張を加えて \cite{j-ikamusume5}
その成果をAjhcとして公開している \cite{ajhc} .
Ajhcは実用化可能なOSを設計可能な型推論をそなえたコンパイラを目指している.
本稿ではAjhcを使ったOSの開発手法を提案し,その手法を小規模ソフトウェアに対して適用/評価する.
最後に今後の研究計画について述べる.

\section{OS開発における既存手法の問題}

ソフトウェア開発について実行時エラーの削減は重要なテーマである.
エラー削減には設計後のテストや検証など多くの手法があり,
その一つとして設計時に強い型付けの言語を用いる手法も提案されている.
一方LinuxやBSDのような実用化されたOSの開発において
主に使われているプログラミング言語はC言語である.
C言語はMLやHaskellのような型推論をそなえた言語より弱い型付けであるために
しばしば実行時エラーを引き起こす.
アプリケーション領域ではこのC言語は積極的に使われることはなく,
より強い型付けの言語が実用化設計に応用されている.
研究レベルにおいて,
型推論をそなえた強い型付け言語でOSを設計する試みは複数存在する
\cite{funk} \cite{snowflake-os} \cite{house} .
しかしこれらのOSはLinuxやBSDのようにデスクトップ/サーバ用途として実用化されていない.
筆者らは上記のOSが実用化されていない原因は主に3つに大別できると考える.

一つ目は,実用化に辿り着くまでプロジェクト参加者の気力が継続できないということである.
実用化されたLinuxのようなOSはLinux kernelの動作するデスクトップでLinux kernel自体を開発している.
彼等はまた開発以外の日常のデスクトップ用途にもLinux kernelを使用する.
このような開発スタイルはドッグフードと属に呼ばれている.
この開発サイクルで自然に実用的なテストが行なわれ,OSの品質は向上する.
OS開発ではこのドッグフード開発にすばやく辿り着く方法を採用する必要がある.

二つ目は,ハードウェア割込をポーリングで検出していることである.
型推論を持つ言語実装の多くは再入可能ではない.
そのためハードウェア割込は言語のランタイムで受け取り,
OS実装側は定期的にランタイムにためられたイベント通知をラ引き上げる必要がある.
実用化されたOSのほとんどは割込をイベントドリブンでOS実装が直接引き上げる.
つまりUNIX誕生から長年つちかわれたOS設計のノウハウを捨てて,
まったく新しい設計検討をドッグフード開発を開始するまで継続しなければならない.

最後に,これらのOSはC言語で設計された既存のデバイスドライバと共存することができない.
OSの存在意義はアプリケーションを動作させるためであるが,
そのためにはコンピュータに接続されたデバイスを抽象化してアプリケーション側に見せてやる必要がある.
世界中には様々な種類のデバイスが存在し,それぞれに異なるドライバ実装が必要である.
これらのOSではドッグフード開発に必要なドライバ群をすべてゼロから再実装する必要がある.

\section{本稿で提案する開発手法}

xxx スナッチ設計

xxx jhcを選択した理由

\begin{table}[tb]
\caption{``hoge''と印字するプログラムに見るコンパイラの特性}
\label{tab:compilerlist}
\hbox to\hsize{\hfil
\begin{tabular}{l|lll}\hline\hline
コンパイラ実装	& サイズ & 未定義シンボル & 依存ライブラリ \\\hline
GHC-7.4.1	& 797228 B & 144 個 & 9 個 \\
SML\#-1.2.0	& 813460 B & 134 個 & 7 個 \\
OCaml-4.00.1	& 183348 B & 84  個 & 5 個 \\
MLton-20100608	& 170061 B & 71  個 & 5 個 \\
jhc-0.8.0	& 21248 B  & 20  個 & 3 個 \\\hline
\end{tabular}\hfil}
\end{table}

xxx Ajhcでの改造項目: 再入可能に

\section{開発手法の評価と考察}

xxx 適用例: OSなし動作,ChibiOS/RTによるスレッド,Android NDK

\section{結論と今後の課題}

xxx わかったこと: メモリ管理さえあればHaskellは動く,省メモリ化は可能,C言語とHaskellの共存

xxx まだわかっていないこと: コンテキスト間状態共有,メモリの局所書き換え,大規模設計が可能か

xxx 今後の計画: vector/arrayライブラリ移植,NetBSD kernelのスナッチ,ATS2 \cite{ats} の調査

\begin{acknowledgment}
偉大なHaskellコンパイラを産み出し,
誰もが望みを捨ててしまっていたOS領域にかすかな光をもたらしたJohn Meachamに感謝する.
\end{acknowledgment}

% BibTeX %%%%%%%%%%%%%%%%%%%%%%%%%%%%%%%%%
\bibliographystyle{ipsjunsrt}
\bibliography{../bibtex/reference,../bibtex/jreference}

% Biography %%%%%%%%%%%%%%%%%%%%%%%%%%%%%%%%%
\begin{biography}
\profile{n}{岡部 究}{}
\profile{n}{Hiroki MIZUNO}{}
\profile{n}{瀬川 秀一}{}
\end{biography}

\end{document}
