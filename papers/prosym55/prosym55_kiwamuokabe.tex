%%
%% 研究報告用スイッチ
%% [techrep]
%%
%% 欧文表記無しのスイッチ(etitle,jkeyword,eabstract,ekeywordは任意)
%% [noauthor]
%%

%\documentclass[submit,techreq]{ipsj}
\documentclass[submit,techreq,noauthor]{ipsj}


\usepackage[dvips]{graphicx}
\usepackage{latexsym}

\def\Underline{\setbox0\hbox\bgroup\let\\\endUnderline}
\def\endUnderline{\vphantom{y}\egroup\smash{\underline{\box0}}\\}
\def\|{\verb|}


\begin{document}

% Title %%%%%%%%%%%%%%%%%%%%%%%%%%%%%%%%%
\title{強い型によるOSの開発手法の提案}

\affiliate{METASEPI}{Metasepi Project}
\affiliate{OCAML-NAGOYA}{ocaml-nagoya}

\author{岡部 究}{Kiwamu Okabe}{METASEPI}[kiwamu@debian.or.jp]
\author{水野 洋樹}{Hiroki MIZUNO}{OCAML-NAGOYA}[mzp@ocaml.jp]
\author{瀬川 秀一}{Hidekazu SEGAWA}{}[]

\begin{abstract}
現在でも実用化されているOSはC言語によって設計されている.
一方アプリケーション領域ではMLやHaskell言語のような型推論を持つ言語を用いて
安全な設計をする手法が確立されている.
本稿では強い型を使った安全な設計手法をOSのような低レベルのソフトウェアに適用する方法を提案する.
\end{abstract}

\begin{jkeyword}
プログラミング・シンポジウム,プログラミング言語,コンパイラ,Haskell,OS
\end{jkeyword}

\maketitle

% Body %%%%%%%%%%%%%%%%%%%%%%%%%%%%%%%%%
\section{はじめに}

xxx 本稿の概要を簡単に

\subsection{アウトライン xxx}

筆者らはjhc Haskellコンパイラに組込み向け拡張を加えて \cite{j-ikamusume5}
その成果をAjhc Haskellコンパイラとして公開している \cite{ajhc} .

\section{既存手法の問題}

xxx 割り込みを受けられないために,既存OSの設計を転用できない

xxx ドッグフードに辿り着く前に息絶える

xxx 移植性

xxx C言語デバイスドライバとの共存

\section{本稿で提案する開発手法}

xxx jhcを選択した理由

xxx Ajhcでの改造項目

xxx スナッチ設計

\section{開発手法の評価と考察}

xxx 適用例: OSなし動作,ChibiOS/RTによるスレッド,Android NDK

\section{結論と今後の課題}

\begin{acknowledgment}
偉大なHaskellコンパイラを産み出し,
誰もが望みを捨ててしまっていたOS領域にかすかな光をもたらしたJohn Meachamに感謝する.
\end{acknowledgment}

% BibTeX %%%%%%%%%%%%%%%%%%%%%%%%%%%%%%%%%
\bibliographystyle{ipsjunsrt}
\bibliography{../bibtex/reference,../bibtex/jreference}

% Biography %%%%%%%%%%%%%%%%%%%%%%%%%%%%%%%%%
\begin{biography}
\profile{n}{岡部 究}{}
\profile{n}{Hiroki MIZUNO}{}
\profile{n}{瀬川 秀一}{}
\end{biography}

\end{document}
